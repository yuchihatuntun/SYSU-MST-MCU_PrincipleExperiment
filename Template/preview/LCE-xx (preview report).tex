% 若编译失败,且生成 .synctex(busy) 辅助文件,可能有两个原因:
% 1. 需要插入的图片不存在:Ctrl + F 搜索 'figure' 将这些代码注释/删除掉即可
% 2. 路径/文件名含中文或空格:更改路径/文件名即可

% --------------------- 文章宏包及相关设置 --------------------- %
% >> ------------------ 文章宏包及相关设置 ------------------ << %
% 设定文章类型与编码格式
\documentclass[UTF8]{article}		
\input{../../.config/config_for_LinearCircuitExperiment.tex}



%%%%%%%%%%%%%%%%%%%%%%%%%%%%%%%%%%%%%%%%%%%%%%%%%%%%%%%%%%%%%%%%
% 仅需修改页眉、实验名称、实验日期
%%%%%%%%%%%%%%%%%%%%%%%%%%%%%%%%%%%%%%%%%%%%%%%%%%%%%%%%%%%%%%%%


%%%%%%%%%%%%%%%%%% 1. 修改页眉内容 %%%%%%%%%%%%%%%%%%
\rhead{Preview Report of LCE-xx xxxxxxxx (2025.xx.xx, 丁毅)}

% 开始编辑文章
\begin{document}
\begin{center}\large
    \vspace*{-0.8cm}
    \noindent{\huge\bfseries《\ \ 线\ \ 性\ \ 电\ \ 路\ \ 实\ \ 验\ \ \ 》\ \ 预\ \ 习\ \ 报\ \ 告 }
    \\\vspace{0.1cm}
    \noindent{
    {\bfseries 
%
%%%%%%%%%%%%%%%%%% 2. 修改实验名称 %%%%%%%%%%%%%%%%%%
    实验名称:\uline{\hspace{2.2cm} xxxxxx \hspace{2.2cm}}
%
    }\hspace{0.4cm}
    指导教师:\uline{\hspace{0.8cm}王东雷\ \ \ \  \ df4dac@sina.com     \hspace{0.8cm}}
    }
    \\\vspace{0.1cm}
    \noindent
    {
    姓名:\uline{\,\,\,丁毅\,\,\,}\hspace{0.2cm}
    学号:\uline{\,\,\,{ 2023K8009908031}\,\,\,}\hspace{0.2cm}
    班级/专业:\uline{\,\,\,{2308/电子信息}\,\,\,}\hspace{0.2cm}
    分组序号:\uline{\,\,\,{2-06}\,\,\,}
    }
    \\\vspace{0.1cm}
    \noindent{
%
%%%%%%%%%%%%%%%%%% 3. 修改实验日期 %%%%%%%%%%%%%%%%%%
    实验日期:\uline{\,\,{2025.xx.xx}\,\,}\hspace{0.2cm}
%
    实验地点:\uline{\,\,\,教学楼{ 607}\,\,\,}\hspace{0.2cm}
    是否调课/补课:\uline{\hspace{0.26cm}否 \hspace{0.26cm}}\hspace{0.2cm}
    成绩:\uline{\hspace{2cm}}
    }
\end{center}
\vspace{-0.4cm}
\noindent\rule{\textwidth}{0.075em}   % 分割线
\vspace{-1.0cm}


% ------------------------ 文章信息区 ------------------------ %
% ------------------------ 文章信息区 ------------------------ %



%%%%%%%%%%%%%%%%%%%%%%%%%%%%%%%%%%%%%%%%%%%%%%%%%%%%%%%%%%%%%%%%%%%%%%%%%%%%%%%%%
%%%%%%%%%%%%%%%%%%%%%%%%%%%%%%%%% 下面是正文内容 %%%%%%%%%%%%%%%%%%%%%%%%%%%%%%%%%
%%%%%%%%%%%%%%%%%%%%%%%%%%%%%%%%%%%%%%%%%%%%%%%%%%%%%%%%%%%%%%%%%%%%%%%%%%%%%%%%%

\section{实验目的}



\section{实验仪器}

\section{实验内容}

%\section{异常现象分析}
















































\end{document}

% VScode 常用快捷键:

% F2:                       变量重命名
% Ctrl + Enter:             行中换行
% Alt + up/down:            上下移行
% 鼠标中键 + 移动:           快速多光标
% Shift + Alt + up/down:    上下复制
% Ctrl + left/right:        左右跳单词
% Ctrl + Backspace/Delete:  左右删单词    
% Shift + Delete:           删除此行
% Ctrl + J:                 打开 VScode 下栏(输出栏)
% Ctrl + B:                 打开 VScode 左栏(目录栏)
% Ctrl + `:                 打开 VScode 终端栏
% Ctrl + 0:                 定位文件
% Ctrl + Tab:               切换已打开的文件(切标签)
% Ctrl + Shift + P:         打开全局命令(设置)

% Latex 常用快捷键:

% Ctrl + Alt + J:           由代码定位到PDF


