% 若编译失败,且生成 .synctex(busy) 辅助文件,可能有两个原因:
% 1. 需要插入的图片不存在:Ctrl + F 搜索 'figure' 将这些代码注释/删除掉即可
% 2. 路径/文件名含中文或空格:更改路径/文件名即可

% --------------------- 文章宏包及相关设置 --------------------- %
% >> ------------------ 文章宏包及相关设置 ------------------ << %
% 设定文章类型与编码格式
\documentclass[UTF8]{article}		

% 物理实验报告所需的其它宏包
\usepackage{ulem}   % \uline 下划线支持
\usepackage{circuitikz} % 电路图 tikz 支持
\usepackage{pdfpages}   % 用于导入 pdf 文件
\usepackage{makecell}   % 用于表格中的换行和合并单元格

% 本 .tex 专属的宏定义
    \def\V{\ \mathrm{V}}
    \def\uV{\ \mu\mathrm{V}}
    \def\mV{\ \mathrm{mV}}
    \def\K{\ \mathrm{K}}
    \def\kV{\ \mathrm{KV}}
    \def\KV{\ \mathrm{KV}}
    \def\MV{\ \mathrm{MV}}
    \def\uA{\ \mu\mathrm{A}}
    \def\mA{\ \mathrm{mA}}
    \def\A{\ \mathrm{A}}
    \def\kA{\ \mathrm{KA}}
    \def\KA{\ \mathrm{KA}}
    \def\MA{\ \mathrm{MA}}
    \def\O{\ \Omega}
    \def\mO{\ \Omega}
    \def\kO{\ \mathrm{K}\Omega}
    \def\KO{\ \mathrm{K}\Omega}
    \def\MO{\ \mathrm{M}\Omega}
    \def\Hz{\ \mathrm{Hz}}
    \def\uF{\ \mu\mathrm{F}}
    \def\mF{\ \mathrm{mF}}
    \def\F{\ \mathrm{F}}
    \def\Re{\mathrm{\,Re}\,}
    \def\Im{\mathrm{\,Im}\,}
    \def\sinc{\mathrm{\,sinc}\,}

% 自定义宏定义
    \def\N{\mathbb{N}}
    \def\F{\mathbb{F}}
    \def\Z{\mathbb{Z}}
    \def\Q{\mathbb{Q}}
    \def\R{\mathbb{R}}
    \def\C{\mathbb{C}}
    \def\T{\mathbb{T}}
    \def\S{\mathbb{S}}
    %\def\A{\mathbb{A}}
    \def\I{\mathscr{I}}
    \def\d{\mathrm{d}}
    \def\p{\partial}


% 导入基本宏包
    \usepackage[UTF8]{ctex}     % 设置文档为中文语言
    \usepackage{hyperref}  % 宏包:自动生成超链接 (此宏包与标题中的数学环境冲突)
    \hypersetup{
        colorlinks=true,    % false:边框链接 ; true:彩色链接
        citecolor={blue},    % 文献引用颜色
        linkcolor={blue},   % 目录 (我们在目录处单独设置),公式,图表,脚注等内部链接颜色
        urlcolor={magenta},    % 网页 URL 链接颜色,包括 \href 中的 text
        % cyan 浅蓝色 
        % magenta 洋红色
        % yellow 黄色
        % black 黑色
        % white 白色
        % red 红色
        % green 绿色
        % blue 蓝色
        % gray 灰色
        % darkgray 深灰色
        % lightgray 浅灰色
        % brown 棕色
        % lime 石灰色
        % olive 橄榄色
        % orange 橙色
        % pink 粉红色
        % purple 紫色
        % teal 蓝绿色
        % violet 紫罗兰色
    }
    % \usepackage{docmute}    % 宏包:子文件导入时自动去除导言区,用于主/子文件的写作方式,\include{./51单片机笔记}即可。注:启用此宏包会导致.tex文件capacity受限。
    \usepackage{amsmath}    % 宏包:数学公式
    \usepackage{mathrsfs}   % 宏包:提供更多数学符号
    \usepackage{amssymb}    % 宏包:提供更多数学符号
    \usepackage{pifont}     % 宏包:提供了特殊符号和字体
    \usepackage{extarrows}  % 宏包:更多箭头符号 
    \usepackage{multicol}   % 宏包:支持多栏 

% 文章页面margin设置
    \usepackage[a4paper]{geometry}
        \geometry{top=0.75in}
        \geometry{bottom=0.75in}
        \geometry{left=0.75in}
        \geometry{right=0.75in}   % 设置上下左右页边距
        \geometry{marginparwidth=1.75cm}    % 设置边注距离(注释、标记等)

% 配置数学环境
    \usepackage{amsthm} % 宏包:数学环境配置
    % theorem-line 环境自定义
        \newtheoremstyle{MyLineTheoremStyle}% <name>
            {11pt}% <space above>
            {11pt}% <space below>
            {\kaishu}% <body font> 默认使用正文字体, \kaishu 为楷体
            {}% <indent amount>
            {\bfseries}% <theorem head font> 设置标题项为加粗
            {:\ \ }% <punctuation after theorem head>
            {.5em}% <space after theorem head>
            {\textbf{#1}\thmnumber{#2}\ \ (\,\textbf{#3}\,)}% 设置标题内容顺序
        \theoremstyle{MyLineTheoremStyle} % 应用自定义的定理样式
        \newtheorem{LineTheorem}{Theorem.\,}
    % theorem-block 环境自定义
        \newtheoremstyle{MyBlockTheoremStyle}% <name>
            {11pt}% <space above>
            {11pt}% <space below>
            {\kaishu}% <body font> 使用默认正文字体
            {}% <indent amount>
            {\bfseries}% <theorem head font> 设置标题项为加粗
            {:\\ \indent}% <punctuation after theorem head>
            {.5em}% <space after theorem head>
            {\textbf{#1}\thmnumber{#2}\ \ (\,\textbf{#3}\,)}% 设置标题内容顺序
        \theoremstyle{MyBlockTheoremStyle} % 应用自定义的定理样式
        \newtheorem{BlockTheorem}[LineTheorem]{Theorem.\,} % 使用 LineTheorem 的计数器
    % definition 环境自定义
        \newtheoremstyle{MySubsubsectionStyle}% <name>
            {11pt}% <space above>
            {11pt}% <space below>
            {}% <body font> 使用默认正文字体
            {}% <indent amount>
            {\bfseries}% <theorem head font> 设置标题项为加粗
            {:\\ \indent}% <punctuation after theorem head>
            {0pt}% <space after theorem head>
            {\textbf{#3}}% 设置标题内容顺序
        \theoremstyle{MySubsubsectionStyle} % 应用自定义的定理样式
        \newtheorem{definition}{}

%宏包:有色文本框(proof环境)及其设置
\usepackage{xcolor}    %设置插入的文本框颜色
    % rgb(4, 9, 103), rgb(5, 13, 164)
    % rgb(124, 131, 255), rgb(231, 232, 255)
    % rgb(255, 190, 190), rgb(255, 70, 70)
    \definecolor{stc}{RGB}{4, 10, 118}  % 设置各级标题结构颜色
\usepackage[strict]{changepage}     % 提供一个 adjustwidth 环境
\usepackage{framed}     % 实现方框效果
    % rgb(0, 0, 0), rgb(100, 100, 100)
    %#ECECED 为 0.93, 0.93, 0.93
    \definecolor{graybox_color}{rgb}{0.93, 0.93, 0.93} % 这里的 rbg 范围是 [0, 1]
    % 文本框颜色。修改此行中的 rgb 数值即可改变方框纹颜色,具体颜色的rgb数值可以在网站https://colordrop.io/ 中获得。(截止目前的尝试还没有成功过,感觉单位不一样)(找到喜欢的颜色,点击下方的小眼睛,找到rgb值,复制修改即可)
    \newenvironment{graybox}{%
    \def\FrameCommand{%
    \hspace{1pt}%
    {\color{gray}\small \vrule width 2pt}%
    {\color{graybox_color}\vrule width 4pt}%
    \colorbox{graybox_color}%
    }%
    \MakeFramed{\advance\hsize-\width\FrameRestore}%
    \noindent\hspace{-4.55pt}% disable indenting first paragraph
    \begin{adjustwidth}{}{7pt}%
    \vspace{2pt}\vspace{2pt}%
    }
    {%
    \vspace{2pt}\end{adjustwidth}\endMakeFramed%
    }

    \definecolor{bluebox_ruleColor}{rgb}{0.49, 0.51, 1} % 文本框颜色。修改此行中的 rgb 数值即可改变方框纹颜色,具体颜色的rgb数值可以在网站https://colordrop.io/ 中获得。(截止目前的尝试还没有成功过,感觉单位不一样)(找到喜欢的颜色,点击下方的小眼睛,找到rgb值,复制修改即可)
    \definecolor{bluebox_backgroundColor}{rgb}{0.93, 0.93, 1}
    \newenvironment{bluebox}{%
    \def\FrameCommand{%
    \hspace{1pt}%
    {\color{bluebox_ruleColor}\small \vrule width 2pt}%
    {\color{bluebox_backgroundColor}\vrule width 4pt}% 4pt 缩进比较合适
    \colorbox{bluebox_backgroundColor}%
    }%
    \MakeFramed{\advance\hsize-\width\FrameRestore}%
    \noindent\hspace{-4.55pt}% disable indenting first paragraph
    \begin{adjustwidth}{}{7pt}%
    \vspace{2pt}\vspace{2pt}%
    }
    {%
    \vspace{2pt}\end{adjustwidth}\endMakeFramed%
    }

    \definecolor{redbox_ruleColor}{rgb}{1, 0.27, 0.27} % 文本框颜色。修改此行中的 rgb 数值即可改变方框纹颜色,具体颜色的rgb数值可以在网站https://colordrop.io/ 中获得。(截止目前的尝试还没有成功过,感觉单位不一样)(找到喜欢的颜色,点击下方的小眼睛,找到rgb值,复制修改即可)
    \definecolor{redbox_backgroundColor}{rgb}{1, 0.90, 0.90}
    \newenvironment{redbox}{%
    \def\FrameCommand{%
    \hspace{1pt}%
    {\color{redbox_ruleColor}\small \vrule width 2pt}%
    {\color{redbox_backgroundColor}\vrule width 4pt}% 4pt 缩进比较合适
    \colorbox{redbox_backgroundColor}%
    }%
    \MakeFramed{\advance\hsize-\width\FrameRestore}%
    \noindent\hspace{-4.55pt}% disable indenting first paragraph
    \begin{adjustwidth}{}{7pt}%
    \vspace{2pt}\vspace{2pt}%
    }
    {%
    \vspace{2pt}\end{adjustwidth}\endMakeFramed%
    }

% 外源代码插入设置
    % matlab 代码插入设置
    \usepackage{matlab-prettifier}
        \lstset{style=Matlab-editor}    % 继承 matlab 代码高亮 , 此行不能删去
    \usepackage[most]{tcolorbox} % 引入tcolorbox包 
    \usepackage{listings} % 引入listings包
        \tcbuselibrary{listings, skins, breakable}
        \newfontfamily\codefont{Consolas} % 定义需要的 codefont 字体
        \lstdefinestyle{MatlabStyle_inc}{   % 插入代码的样式
            language=Matlab,
            basicstyle=\footnotesize\ttfamily\codefont,    % ttfamily 确保等宽 
            breakatwhitespace=false,
            breaklines=true,
            captionpos=b,
            keepspaces=true,
            numbers=left,
            numbersep=15pt,
            showspaces=false,
            showstringspaces=false,
            showtabs=false,
            tabsize=2,
            xleftmargin=15pt,   % 左边距
            %frame=single, % single 为包围式单线框
            frame=shadowbox,    % shadowbox 为带阴影包围式单线框效果
            %escapeinside=``,   % 允许在代码块中使用 LaTeX 命令 (此行无用)
            %frameround=tttt,    % tttt 表示四个角都是圆角
            framextopmargin=0pt,    % 边框上边距
            framexbottommargin=0pt, % 边框下边距
            framexleftmargin=5pt,   % 边框左边距
            framexrightmargin=5pt,  % 边框右边距
            rulesepcolor=\color{red!20!green!20!blue!20}, % 阴影框颜色设置
            %backgroundcolor=\color{blue!10}, % 背景颜色
        }
        \lstdefinestyle{MatlabStyle_src}{   % 插入代码的样式
            language=Matlab,
            basicstyle=\small\ttfamily\codefont,    % ttfamily 确保等宽 
            breakatwhitespace=false,
            breaklines=true,
            captionpos=b,
            keepspaces=true,
            numbers=left,
            numbersep=15pt,
            showspaces=false,
            showstringspaces=false,
            showtabs=false,
            tabsize=2,
        }
        \newtcblisting{matlablisting}{
            %arc=2pt,        % 圆角半径
            % 调整代码在 listing 中的位置以和引入文件时的格式相同
            top=0pt,
            bottom=0pt,
            left=-5pt,
            right=-5pt,
            listing only,   % 此句不能删去
            listing style=MatlabStyle_src,
            breakable,
            colback=white,   % 选一个合适的颜色
            colframe=black!0,   % 感叹号后跟不透明度 (为 0 时完全透明)
        }
        \lstset{
            style=MatlabStyle_inc,
        }

% table 支持
    \usepackage{booktabs}   % 宏包:三线表
    \usepackage{tabularray} % 宏包:表格排版
    \usepackage{longtable}  % 宏包:长表格

% figure 设置
    \usepackage{graphicx}  % 支持 jpg, png, eps, pdf 图片 
    \usepackage{svg}       % 支持 svg 图片
        \svgsetup{
            % 指向 inkscape.exe 的路径
            inkscapeexe = C:/aa_MySame/inkscape/bin/inkscape.exe, 
            % 一定程度上修复导入后图片文字溢出几何图形的问题
            inkscapelatex = false                 
        }
    \usepackage{subcaption} % 用于子图和小图注  

% 图表进阶设置
    \usepackage{caption}    % 图注、表注
        \captionsetup[figure]{name=Figure }  
        \captionsetup[table]{name=Table }
        \captionsetup{
            labelfont=bf, % 设置标签为粗体
            textfont=bf,  % 设置文本为粗体
            font=small  
        }
    \usepackage{float}     % 图表位置浮动设置 
    \usepackage{etoolbox} % 用于保证图注表注的数学字符为粗体
        \AtBeginEnvironment{figure}{\boldmath} % 图注中的数学字符为粗体
        \AtBeginEnvironment{table}{\boldmath}  % 表注中的数学字符为粗体
        \AtBeginEnvironment{tabular}{\unboldmath}   % 保证表格中的数学字符不受额外影响

% 圆圈序号自定义
    \newcommand*\circled[1]{\tikz[baseline=(char.base)]{\node[shape=circle,draw,inner sep=0.8pt, line width = 0.03em] (char) {\bfseries #1};}}   % TikZ solution

% 列表设置
    \usepackage{enumitem}   % 宏包:列表环境设置
        \setlist[enumerate]{
            label=(\arabic*) ,   % 设置序号样式为加粗的 (1) (2) (3)
            ref=\arabic*, % 如果需要引用列表项,这将决定引用格式(这里仍然使用数字)
            itemsep=0pt, parsep=0pt, topsep=0pt, partopsep=0pt, leftmargin=3.5em} 
        \setlist[itemize]{itemsep=0pt, parsep=0pt, topsep=0pt, partopsep=0pt, leftmargin=3.5em}
        \newlist{circledenum}{enumerate}{1} % 创建一个新的枚举环境  
        \setlist[circledenum,1]{  
            label=\protect\circled{\arabic*}, % 使用 \arabic* 来获取当前枚举计数器的值,并用 \circled 包装它  
            ref=\arabic*, % 如果需要引用列表项,这将决定引用格式(这里仍然使用数字)
            itemsep=0pt, parsep=0pt, topsep=0pt, partopsep=0pt, leftmargin=3.5em
        }  

% 其它设置
    % 脚注设置
        \renewcommand\thefootnote{\ding{\numexpr171+\value{footnote}}}
    % 参考文献引用设置
        \bibliographystyle{unsrt}   % 设置参考文献引用格式为unsrt
        \newcommand{\upcite}[1]{\textsuperscript{\cite{#1}}}     % 自定义上角标式引用
    % 文章序言设置
        \newcommand{\cnabstractname}{序言}
        \newenvironment{cnabstract}{%
            \par\Large
            \noindent\mbox{}\hfill{\bfseries \cnabstractname}\hfill\mbox{}\par
            \vskip 2.5ex
            }{\par\vskip 2.5ex}

% 文章默认字体设置
    \usepackage{fontspec}   % 宏包:字体设置
        \setmainfont{SimSun}    % 设置中文字体为宋体字体
        \setCJKmainfont[AutoFakeBold=3]{SimSun} % 设置加粗字体为 SimSun 族,AutoFakeBold 可以调整字体粗细
        \setmainfont{Times New Roman} % 设置英文字体为Times New Roman

% 各级标题自定义设置
    \usepackage{titlesec}   
        % section标题自定义设置 
        \titleformat{\section}[hang]{\normalfont\Large\bfseries\boldmath}{\thesection}{8pt}{}
        % subsection 标题自定义设置
        \titleformat{\subsection}[hang]{\normalfont\large\bfseries\boldmath}{\thesubsection}{8pt}{}
        \titlespacing*{\subsection}{0pt}{10pt}{6pt} % 控制上下间距


% --------------------- 文章宏包及相关设置 --------------------- %
% >> ------------------ 文章宏包及相关设置 ------------------ << %




% ------------------------ 文章信息区 ------------------------ %
% ------------------------ 文章信息区 ------------------------ %
% 页眉页脚设置
\usepackage{fancyhdr}   %宏包:页眉页脚设置
    \pagestyle{fancy}
    \fancyhf{}
    \cfoot{\thepage}
    \renewcommand\headrulewidth{1pt}
    \renewcommand\footrulewidth{0pt}
    %\rhead{《线性电路实验》实验报告}    
    \lhead{\small \faGithub\ \href{https://github.com/yuchihatuntun/SYSU-MST-MCU_PrincipleExperiment}{\color{blue} GitHub Link for This Project}}


    \graphicspath{{../}}   % 修改主文件图像路径,使得子文件能够直接使用相对路径,而不是从 assets 开始索引

    \usepackage{fontawesome}    % 宏包:更多符号与图标 (用于插入 GitHub 图标等)



%%%%%%%%%%%%%%%%%%%%%%%%%%%%%%%%%%%%%%%%%%%%%%%%%%%%%%%%%%%%%%%%
% 仅需修改页眉、实验名称、实验日期
%%%%%%%%%%%%%%%%%%%%%%%%%%%%%%%%%%%%%%%%%%%%%%%%%%%%%%%%%%%%%%%%


%%%%%%%%%%%%%%%%%% 1. 修改页眉内容 %%%%%%%%%%%%%%%%%%
\rhead{MCPE-01 Multi-digit 16-bit Hexadecimal Arithmetic Experiment (2025.11.13, 徐睿琳)}

% 开始编辑文章
\begin{document}
\begin{center}\large
    \vspace*{-0.8cm}
    \noindent{\huge\bfseries《\ \ 微\ \ 机\ \ 原\ \ 理\ \ 实\ \ 验\ \ \ 》\ \ 实\ \ 验\ \ 报\ \ 告 }
    \\\vspace{0.1cm}
    \noindent{
    {\bfseries 
%
%%%%%%%%%%%%%%%%%% 2. 修改实验名称 %%%%%%%%%%%%%%%%%%
    实验名称:\uline{\hspace{2.2cm} 多位16进制加法运算实验 \hspace{2.2cm}}
%
    }\hspace{0.4cm}
    指导教师:\uline{\hspace{0.8cm}肖山林     \hspace{0.8cm}}
    }
    \\\vspace{0.1cm}
    \noindent
    {
    姓名:\uline{\,\,\,徐睿琳\,\,\,}\hspace{0.2cm}
    学号:\uline{\,\,\,{ 23342107}\,\,\,}\hspace{0.2cm}
    专业/班级:\uline{\,\,\,{微电子/三班}\,\,\,}\hspace{0.2cm}
    分组序号:\uline{\,\,\,{A412-A05}\,\,\,}
    }
    \\\vspace{0.1cm}
    \noindent{
%
%%%%%%%%%%%%%%%%%% 3. 修改实验日期 %%%%%%%%%%%%%%%%%%
    实验日期:\uline{\,\,{2025.11.13}\,\,}\hspace{0.2cm}
%
    实验地点:\uline{\,\,\,教学楼{ A412}\,\,\,}\hspace{0.2cm}
    是否调课/补课:\uline{\hspace{0.26cm}否 \hspace{0.26cm}}\hspace{0.2cm}
    成绩:\uline{\hspace{2cm}}
    }
\end{center}
\vspace{-0.4cm}
\noindent\rule{\textwidth}{0.075em}   % 分割线
\vspace{-1.0cm}

% 生成目录
\setcounter{tocdepth}{3}  % 目录深度为 2(不显示 subsubsection)
\noindent\tableofcontents\thispagestyle{fancy}   % 显示页码、页眉等

% ------------------------ 文章信息区 ------------------------ %
% ------------------------ 文章信息区 ------------------------ %



%%%%%%%%%%%%%%%%%%%%%%%%%%%%%%%%%%%%%%%%%%%%%%%%%%%%%%%%%%%%%%%%%%%%%%%%%%%%%%%%%
%%%%%%%%%%%%%%%%%%%%%%%%%%%%%%%%% 下面是正文内容 %%%%%%%%%%%%%%%%%%%%%%%%%%%%%%%%%
%%%%%%%%%%%%%%%%%%%%%%%%%%%%%%%%%%%%%%%%%%%%%%%%%%%%%%%%%%%%%%%%%%%%%%%%%%%%%%%%%

\section{实验内容与设计}

\subsection{多位十六进制加法运算实验}

\subsubsection{实验电路图}

\begin{figure}[H]\centering
    \includegraphics[width=0.4\columnwidth]{MCPE-01 Multi-digit 16-bit Hexadecimal Arithmetic Experiment/assets/circuit/circuit.png}
    \caption{实验电路图}
\end{figure}

\subsubsection{实验设计}

\begin{tabular}{lllll}
\toprule
实验 & 实验目的 & \texttt{DATA} 段数据 & \texttt{CODE} 段指令 & 预期标志位结果 \\
\midrule
基础实验 & 观察“无特殊情况”的加法 & 
\begin{tabular}[t]{@{}l@{}}
  \texttt{NUM1 DW 1111H} \\
  \texttt{NUM2 DW 2222H} \\
  \texttt{NUM3 DW 3333H}
\end{tabular} &
\begin{tabular}[t]{@{}l@{}}
  \texttt{ADD AX,[SI+0]} \\
  \texttt{ADD AX,[SI+2]} \\
  \texttt{ADD AX,[SI+4]}
\end{tabular} &
\begin{tabular}[t]{@{}l@{}}
  \texttt{AX = 6666H} \\
  \texttt{CF=0, ZF=0} \\
  \texttt{OF=0, SF=0}
\end{tabular} \\

\midrule
扩展一 & 观察 进位(CF) 和 零(ZF) &
\begin{tabular}[t]{@{}l@{}}
  \texttt{NUM1 DW 0F000H} \\
  \texttt{NUM2 DW 00FFFH} \\
  \texttt{NUM3 DW 00001H}
\end{tabular} &
\texttt{ADD AX,[SI+4]} &
\begin{tabular}[t]{@{}l@{}}
  \texttt{AX = 0000H} \\
  \texttt{CF = 1} (进位) \\
  \texttt{ZF = 1} (结果为零)
\end{tabular} \\

\midrule
扩展二 & 观察 溢出(OF) 和 符号(SF) &
\begin{tabular}[t]{@{}l@{}}
  \texttt{NUM1 DW 6000H} \\
  \texttt{NUM2 DW 0000H} \\
  \texttt{NUM3 DW 6000H}
\end{tabular} &
\texttt{ADD AX,[SI+4]} &
\begin{tabular}[t]{@{}l@{}}
  \texttt{AX = C000H} \\
  \texttt{OF = 1} (溢出) \\
  \texttt{SF = 1} (结果为负)
\end{tabular} \\

\midrule
扩展三 & 观察 OF 和 CF 同时为 1 &
\begin{tabular}[t]{@{}l@{}}
  \texttt{NUM1 DW 8000H} \\
  \texttt{NUM2 DW 0000H} \\
  \texttt{NUM3 DW 8000H}
\end{tabular} &
\texttt{ADD AX,[SI+4]} &
\begin{tabular}[t]{@{}l@{}}
  \texttt{AX = 0000H} \\
  \texttt{CF = 1} (无符号进位) \\
  \texttt{OF = 1} (有符号溢出) \\
  \texttt{ZF = 1}
\end{tabular} \\

\midrule
扩展四 & 实现 \texttt{ADC} (带进位加法) &
\begin{tabular}[t]{@{}l@{}}
  \texttt{NUM1 DW 0FFFFH} \\
  \texttt{NUM2 DW 00001H} \\
  \texttt{NUM3 DW 00001H}
\end{tabular} &
\begin{tabular}[t]{@{}l@{}}
  \texttt{ADD AX,[SI+2]} \\
  \texttt{ADC AX,[SI+4]}
\end{tabular} &
\begin{tabular}[t]{@{}l@{}}
  \texttt{ADD} 后 \texttt{CF=1} \\
  \texttt{ADC} 后 \texttt{AX = 0002H} \\
  (因为 \texttt{0+1+CF})
\end{tabular} \\

\midrule
扩展五 & 实现 \texttt{INC} (不影响CF) &
\begin{tabular}[t]{@{}l@{}}
  \texttt{NUM1 DW 0FFFFH}
\end{tabular} &
\begin{tabular}[t]{@{}l@{}}
  \texttt{MOV AX, [SI+0]} \\
  \texttt{INC AX}
\end{tabular} &
\begin{tabular}[t]{@{}l@{}}
  \texttt{AX = 0000H} \\
  \texttt{ZF = 1} \\
  \texttt{CF = 0} (或保持不变)
\end{tabular} \\
\bottomrule
\end{tabular}

\subsection{多位十六进制减法运算实验}

\subsubsection{实验设计}

\begin{tabular}{lllll}
\toprule
实验 & 实验目的 & \texttt{DATA} 段数据 & \texttt{CODE} 段指令 & 预期标志位结果 \\
\midrule
基础实验 & 实现基础\texttt{SUB}指令,无特殊情况 & 
\begin{tabular}[t]{@{}l@{}}
  \texttt{N1 DW 3333H} \\
  \texttt{N2 DW 1111H}
\end{tabular} &
\begin{tabular}[t]{@{}l@{}}
  \texttt{MOV AX, [N1]} \\
  \texttt{SUB AX, [N2]}
\end{tabular} &
\begin{tabular}[t]{@{}l@{}}
  \texttt{AX = 2222H} \\
  \texttt{CF=0} (无借位) \\
  \texttt{ZF=0, SF=0, OF=0}
\end{tabular} \\

\midrule
扩展一 & \texttt{SUB}指令,相减为零、借位情况 &
\begin{tabular}[t]{@{}l@{}}
  \texttt{N1 DW 5555H} \\
  \texttt{N2 DW 5555H} \\
  \texttt{N3 DW 0001H}
\end{tabular} &
\begin{tabular}[t]{@{}l@{}}
  \texttt{MOV AX, [N1]} \\
  \texttt{SUB AX, [N2]} \\
  \texttt{SUB AX, [N3]}
\end{tabular} &
\begin{tabular}[t]{@{}l@{}}
  第一次 \texttt{SUB} 后: \\
  \texttt{AX = 0000H} \\
  \texttt{ZF = 1} (零标志) \\
  第二次 \texttt{SUB} 后: \\
  \texttt{AX = FFFFH} \\
  \texttt{CF = 1} (借位标志)
\end{tabular} \\

\midrule
扩展二 & \texttt{SUB}溢出情况 &
\begin{tabular}[t]{@{}l@{}}
  \texttt{N1 DW 8000H} \\
  \texttt{N2 DW 0001H}
\end{tabular} &
\begin{tabular}[t]{@{}l@{}}
  \texttt{MOV AX, [N1]} \\
  \texttt{SUB AX, [N2]}
\end{tabular} &
\begin{tabular}[t]{@{}l@{}}
  (负 - 正 = 正) \\
  \texttt{AX = 7FFFH} \\
  \texttt{OF = 1} (溢出标志) \\
  \texttt{SF = 0} (结果为正) \\
  \texttt{CF = 0} (无借位)
\end{tabular} \\

\midrule
扩展三 & 掌握\texttt{SBB}指令 &
\begin{tabular}[t]{@{}l@{}}
  \texttt{N1 DW 1000H} \\
  \texttt{N2 DW 2000H} \\
  \texttt{N3 DW 0001H}
\end{tabular} &
\begin{tabular}[t]{@{}l@{}}
  \texttt{MOV AX, [N1]} \\
  \texttt{SUB AX, [N2]} \\
  \texttt{SBB AX, [N3]}
\end{tabular} &
\begin{tabular}[t]{@{}l@{}}
  \texttt{SUB} 后: \texttt{CF = 1} \\
  \texttt{SBB} 后: \\
  \texttt{AX = AX-N3-CF} \\
  \texttt{AX = F000-1-1} \\
  \texttt{AX = EFFEH}
\end{tabular} \\

\midrule
扩展四 & 掌握\texttt{DEC}指令 &
\begin{tabular}[t]{@{}l@{}}
  \texttt{N1 DW 0000H}
\end{tabular} &
\begin{tabular}[t]{@{}l@{}}
  \texttt{MOV AX, [N1]} \\
  \texttt{DEC AX}
\end{tabular} &
\begin{tabular}[t]{@{}l@{}}
  \texttt{AX = FFFFH} \\
  \texttt{CF = 0} (关键!) \\
  (\texttt{DEC} 不影响 \texttt{CF}) \\
  \texttt{ZF=0, SF=1}
\end{tabular} \\

\midrule
扩展五 & 掌握\texttt{CMP}指令 &
\begin{tabular}[t]{@{}l@{}}
  \texttt{N1 DW 5000H} \\
  \texttt{N2 DW 6000H}
\end{tabular} &
\begin{tabular}[t]{@{}l@{}}
  \texttt{MOV AX, [N1]} \\
  \texttt{CMP AX, [N1]} \\
  \texttt{CMP AX, [N2]}
\end{tabular} &
\begin{tabular}[t]{@{}l@{}}
  第一次 \texttt{CMP} (相等): \\
  \texttt{AX} 不变 (\texttt{5000H}) \\
  \texttt{ZF = 1}, \texttt{CF = 0} \\
  第二次 \texttt{CMP} (小于): \\
  \texttt{AX} 不变 (\texttt{5000H}) \\
  \texttt{ZF = 0}, \texttt{CF = 1}
\end{tabular} \\

\midrule
扩展六 & 掌握\texttt{NEG}指令 &
\begin{tabular}[t]{@{}l@{}}
  \texttt{N1 DW 8000H}
\end{tabular} &
\begin{tabular}[t]{@{}l@{}}
  \texttt{MOV AX, [N1]} \\
  \texttt{NEG AX}
\end{tabular} &
\begin{tabular}[t]{@{}l@{}}
  (特殊: 最小负数) \\
  \texttt{AX = 8000H} \\
  \texttt{CF = 1} (有借位) \\
  \texttt{OF = 1} (溢出)
\end{tabular} \\
\bottomrule
\end{tabular}

\newpage

\subsection{多位十六进制乘法运算实验}

\subsubsection{实验设计}

\begin{tabular}{l l l l l}
\toprule
实验名称 & 实验目的 & \texttt{DATA} 段数据 & \texttt{CODE} 段指令 & 预期标志位/寄存器结果 \\
\midrule
基础实验 & \texttt{MUL} (8位) & 
\begin{tabular}[t]{@{}l@{}}
  \texttt{N1 DB 10H} (16) \\
  \texttt{N2 DB 05H} (5)
\end{tabular} &
\begin{tabular}[t]{@{}l@{}}
  \texttt{MOV AL, [N1]} \\
  \texttt{MOV BL, [N2]} \\
  \texttt{MUL BL}
\end{tabular} &
\begin{tabular}[t]{@{}l@{}}
  (\texttt{AX = AL * BL}) \\
  \texttt{AX = 0050H} (80) \\
  \texttt{AH = 00H}, 因此: \\
  \texttt{CF = 0}, \texttt{OF = 0}
\end{tabular} \\

\midrule
扩展一 & \texttt{MUL} (16位) &
\begin{tabular}[t]{@{}l@{}}
  \texttt{N1 DW 8000H} \\
  \texttt{N2 DW 0010H}
\end{tabular} &
\begin{tabular}[t]{@{}l@{}}
  \texttt{MOV AX, [N1]} \\
  \texttt{MOV BX, [N2]} \\
  \texttt{MUL BX}
\end{tabular} &
\begin{tabular}[t]{@{}l@{}}
  (\texttt{DX:AX = AX * BX}) \\
  \texttt{AX = 0000H} \\
  \texttt{DX = 0008H} \\
  \texttt{DX != 0}, 因此: \\
  \texttt{CF = 1}, \texttt{OF = 1}
\end{tabular} \\

\midrule
扩展二 & \texttt{IMUL} (8位) &
\begin{tabular}[t]{@{}l@{}}
  \texttt{N1 DB 0FEH} (-2) \\
  \texttt{N2 DB 04H} (+4)
\end{tabular} &
\begin{tabular}[t]{@{}l@{}}
  \texttt{MOV AL, [N1]} \\
  \texttt{MOV BL, [N2]} \\
  \texttt{IMUL BL}
\end{tabular} &
\begin{tabular}[t]{@{}l@{}}
  (\texttt{AX = AL * BL}) \\
  \texttt{AX = FFF8H} (-8) \\
  (\texttt{AH} 是 \texttt{AL} 的符号扩展) \\
  \texttt{CF = 0}, \texttt{OF = 0}
\end{tabular} \\

\midrule
扩展三 & \texttt{IMUL} (16位) &
\begin{tabular}[t]{@{}l@{}}
  \texttt{N1 DW 4000H} (+16384) \\
  \texttt{N2 DW 0003H} (+3)
\end{tabular} &
\begin{tabular}[t]{@{}l@{}}
  \texttt{MOV AX, [N1]} \\
  \texttt{MOV BX, [N2]} \\
  \texttt{IMUL BX}
\end{tabular} &
\begin{tabular}[t]{@{}l@{}}
  (\texttt{DX:AX = AX * BX}) \\
  \texttt{AX = C000H} \\
  \texttt{DX = 0000H} \\
  (结果 \texttt{+49152} 无法
  存入 \texttt{AX}) \\
  \texttt{CF = 1}, \texttt{OF = 1}
\end{tabular} \\

\midrule
扩展四 & \texttt{MUL} vs \texttt{IMUL} &
\begin{tabular}[t]{@{}l@{}}
  \texttt{N1 DW 0FFFFH} \\
  \texttt{N2 DW 0002H}
\end{tabular} &
\begin{tabular}[t]{@{}l@{}}
  \texttt{MOV AX, [N1]} \\
  \texttt{MOV BX, [N2]} \\
\end{tabular} &
\begin{tabular}[t]{@{}l@{}}
  \texttt{MUL} (无符号): \\
  \texttt{65535 * 2 = 131070} \\
  \texttt{DX:AX = 0001:FFFEH} \\
  \texttt{CF=1, OF=1} \\
  \texttt{IMUL} (有符号): \\
  \texttt{-1 * 2 = -2} \\
  \texttt{DX:AX = FFFF:FFFEH} \\
  \texttt{CF=0, OF=0}
\end{tabular} \\
\bottomrule
\end{tabular}

\section{实验结果}

\subsection{多位十六进制加法运算实验}

\subsubsection{基础实验}

\begin{figure}[H]\centering
    \includegraphics[width=\columnwidth]{MCPE-01 Multi-digit 16-bit Hexadecimal Arithmetic Experiment/assets/1.png}
    \caption{\texttt{MOV AX, DATA}后断点}
\end{figure}

\vspace*{-3mm}
\begin{bluebox}
    程序刚刚执行完指令\texttt{MOV AX, DATA},\texttt{DATA} 是数据段的段地址,汇编器分配给它的值是\texttt{0002H}。\texttt{AX} 已经准备好了数据段的起始地址 0002H,但地址尚未交付给 \texttt{DS} 寄存器。需要下一步指令:\texttt{MOV DS,AX} 将 \texttt{AX} 中的 \texttt{0002H} 移入 \texttt{DS},从而建立起正确的数据段寻址环境。
\end{bluebox}

\begin{figure}[H]\centering
    \includegraphics[width=\columnwidth]{MCPE-01 Multi-digit 16-bit Hexadecimal Arithmetic Experiment/assets/2.png}
    \caption{\texttt{MOV SI,OFFSET NUM1}后断点}
\end{figure}

\vspace*{-3mm}
\begin{bluebox}
    \texttt{SI} 寄存器现在存储了变量 \texttt{NUM1} 在数据段内的起始偏移地址 \texttt{0002H},这为后续通过 \texttt{DS:SI} 访问数据做好了源指针准备。
\end{bluebox}

\begin{figure}[H]\centering
\begin{subfigure}[b]{0.5\columnwidth}\centering
    \includegraphics[height=170pt]{MCPE-01 Multi-digit 16-bit Hexadecimal Arithmetic Experiment/assets/3.png}
    \caption{\texttt{ADD AX,[SI+0]}}
\end{subfigure}\hfill
\begin{subfigure}[b]{0.5\columnwidth}\centering
    \includegraphics[height=170pt]{MCPE-01 Multi-digit 16-bit Hexadecimal Arithmetic Experiment/assets/4.png}
    \caption{\texttt{ADD AX,[SI+1]}}
\end{subfigure}
\begin{subfigure}[b]{0.5\columnwidth}\centering
    \includegraphics[height=170pt]{MCPE-01 Multi-digit 16-bit Hexadecimal Arithmetic Experiment/assets/5.png}
    \caption{\texttt{ADD AX,[SI+2]}}
\end{subfigure}
\caption{加法运算过程}
\end{figure}

\vspace*{-3mm}
\begin{bluebox}
    如图可以看到\texttt{AX}寄存器的数值逐步累加,最终结果为\texttt{6666H},符合预期。同时,标志寄存器中的PF被置位,表示结果为偶数个1,这与\texttt{6666H}的二进制表示相符。
\end{bluebox}

\begin{figure}[H]\centering
    \includegraphics[width=\columnwidth]{MCPE-01 Multi-digit 16-bit Hexadecimal Arithmetic Experiment/assets/6.png}
    \caption{\texttt{MOV [SI+6],AX}后断点}
\end{figure}

\vspace*{-3mm}
\begin{bluebox}
    如图可以看到最终结果\texttt{6666H}已经成功存储在变量\texttt{NUM4}对应的内存地址中,验证了加法运算的正确性。
\end{bluebox}

\subsubsection{扩展一}

\begin{redbox}
    注意:由于部分扩展实验为课后本地实验,Debug.exe无法直接运行,因此以下扩展实验的结果均通过\href{https://github.com/YJDoc2/8086-Emulator}{\color{blue} 8086 Emulator} 仿真软件进行验证,截图均来自该软件。且由于运行环境不同,部分代码与预期代码略有差异,但均实现了相同的功能。
\end{redbox}

\begin{figure}[H]\centering
    \includegraphics[width=\columnwidth]{MCPE-01 Multi-digit 16-bit Hexadecimal Arithmetic Experiment/assets/7.png}
    \caption{\texttt{F000H + 0FFFH}后断点}
\end{figure}

\begin{bluebox}
    如图可以看到SF和PF被置位,表示结果为负数且二进制表示中有偶数个1。
\end{bluebox}

\begin{figure}[H]\centering
    \includegraphics[width=\columnwidth]{MCPE-01 Multi-digit 16-bit Hexadecimal Arithmetic Experiment/assets/8.png}
    \caption{\texttt{FFFFH + 0001H}后断点}
\end{figure}

\section{数据分析与思考总结}

% \subsection*{4.1 \ \ }
% \subsection*{4.2 \ \ }
% \subsection*{4.3 \ \ }
% \subsection*{4.4 \ \ }
% \subsection*{4.5 \ \ }
% \subsection*{4.6 \ \ }




















































\newpage
% 附录
\section*{附录\hspace*{20pt} 实验汇编代码}
\addcontentsline{toc}{section}{附录 A\hspace*{6pt} 实验汇编代码} 
\thispagestyle{fancy} 

\begin{lstlisting}[language={[x86masm]Assembler}, caption={加法运算-基础实验}]
CODE SEGMENT
ASSUME CS:CODE,DS:DATA
BEG: 	MOV AX,DATA
	MOV DS,AX
	MOV SI,OFFSET NUM1
        MOV AX,0
        ADD AX,[SI+0]
        ADD AX,[SI+2]
        ADD AX,[SI+4]   
        MOV [SI+6],AX
	JMP $
CODE ENDS
DATA SEGMENT
NUM1 DW 1111H ;N1
NUM2 DW 2222H ;N2
NUM3 DW 3333H ;N3
NUM4 DW 0000H ;N4
DATA ENDS
	END BEG
\end{lstlisting}

\begin{lstlisting}[language={[x86masm]Assembler}, caption={加法运算-扩展一}]
OPR1: DW 0x0000         
OPR2: DW 0xF000         
OPR3: DW 0x0FFF         
OPR4: DW 0x0001         
RESULT: DW 0            

start:
MOV AX, word OPR1       
MOV BX, word OPR2       
CLC                     
ADD AX, BX              

MOV BX, word OPR3       
ADD AX, BX              

MOV BX, word OPR4       
ADD AX, BX              

MOV DI, OFFSET RESULT   
MOV word [DI], AX       
print reg               
\end{lstlisting}

\begin{lstlisting}[language={[x86masm]Assembler}, caption={加法运算-扩展二}]
CODE SEGMENT
ASSUME CS:CODE
BEG: 	
	MOV AX, 0      
	ADD AX, 6000H 
	ADD AX, 6000H 
	JMP $          
CODE ENDS
	END BEG
\end{lstlisting}

\begin{lstlisting}[language={[x86masm]Assembler}, caption={加法运算-扩展三}]
CODE SEGMENT
ASSUME CS:CODE
BEG: 	
	MOV AX, 0      
	ADD AX, 8000H 
	ADD AX, 8000H 
	JMP $          
CODE ENDS
	END BEG
\end{lstlisting}

\begin{lstlisting}[language={[x86masm]Assembler}, caption={加法运算-扩展四}]
CODE SEGMENT
ASSUME CS:CODE
BEG: 	
	MOV AX, FFFFH    
	ADD AX, 0001H 
  ADC AX, 0001H 
	JMP $          
CODE ENDS
	END BEG
\end{lstlisting}

\begin{lstlisting}[language={[x86masm]Assembler}, caption={加法运算-扩展五}]
CODE SEGMENT
ASSUME CS:CODE
BEG: 	
	MOV AX, FFFFH    
	INC AX
	JMP $          
CODE ENDS
	END BEG
\end{lstlisting}

\end{document}

% VScode 常用快捷键:

% F2:                       变量重命名
% Ctrl + Enter:             行中换行
% Alt + up/down:            上下移行
% 鼠标中键 + 移动:           快速多光标
% Shift + Alt + up/down:    上下复制
% Ctrl + left/right:        左右跳单词
% Ctrl + Backspace/Delete:  左右删单词    
% Shift + Delete:           删除此行
% Ctrl + J:                 打开 VScode 下栏(输出栏)
% Ctrl + B:                 打开 VScode 左栏(目录栏)
% Ctrl + `:                 打开 VScode 终端栏
% Ctrl + 0:                 定位文件
% Ctrl + Tab:               切换已打开的文件(切标签)
% Ctrl + Shift + P:         打开全局命令(设置)

% Latex 常用快捷键:

% Ctrl + Alt + J:           由代码定位到PDF


